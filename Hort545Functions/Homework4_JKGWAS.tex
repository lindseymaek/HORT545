% Options for packages loaded elsewhere
\PassOptionsToPackage{unicode}{hyperref}
\PassOptionsToPackage{hyphens}{url}
%
\documentclass[
]{article}
\usepackage{lmodern}
\usepackage{amssymb,amsmath}
\usepackage{ifxetex,ifluatex}
\ifnum 0\ifxetex 1\fi\ifluatex 1\fi=0 % if pdftex
  \usepackage[T1]{fontenc}
  \usepackage[utf8]{inputenc}
  \usepackage{textcomp} % provide euro and other symbols
\else % if luatex or xetex
  \usepackage{unicode-math}
  \defaultfontfeatures{Scale=MatchLowercase}
  \defaultfontfeatures[\rmfamily]{Ligatures=TeX,Scale=1}
\fi
% Use upquote if available, for straight quotes in verbatim environments
\IfFileExists{upquote.sty}{\usepackage{upquote}}{}
\IfFileExists{microtype.sty}{% use microtype if available
  \usepackage[]{microtype}
  \UseMicrotypeSet[protrusion]{basicmath} % disable protrusion for tt fonts
}{}
\makeatletter
\@ifundefined{KOMAClassName}{% if non-KOMA class
  \IfFileExists{parskip.sty}{%
    \usepackage{parskip}
  }{% else
    \setlength{\parindent}{0pt}
    \setlength{\parskip}{6pt plus 2pt minus 1pt}}
}{% if KOMA class
  \KOMAoptions{parskip=half}}
\makeatother
\usepackage{xcolor}
\IfFileExists{xurl.sty}{\usepackage{xurl}}{} % add URL line breaks if available
\IfFileExists{bookmark.sty}{\usepackage{bookmark}}{\usepackage{hyperref}}
\hypersetup{
  pdftitle={Homework4\_JKGWAS},
  pdfauthor={Pabitra Joshi and Lindsey Kornowske},
  hidelinks,
  pdfcreator={LaTeX via pandoc}}
\urlstyle{same} % disable monospaced font for URLs
\usepackage[margin=1in]{geometry}
\usepackage{color}
\usepackage{fancyvrb}
\newcommand{\VerbBar}{|}
\newcommand{\VERB}{\Verb[commandchars=\\\{\}]}
\DefineVerbatimEnvironment{Highlighting}{Verbatim}{commandchars=\\\{\}}
% Add ',fontsize=\small' for more characters per line
\usepackage{framed}
\definecolor{shadecolor}{RGB}{248,248,248}
\newenvironment{Shaded}{\begin{snugshade}}{\end{snugshade}}
\newcommand{\AlertTok}[1]{\textcolor[rgb]{0.94,0.16,0.16}{#1}}
\newcommand{\AnnotationTok}[1]{\textcolor[rgb]{0.56,0.35,0.01}{\textbf{\textit{#1}}}}
\newcommand{\AttributeTok}[1]{\textcolor[rgb]{0.77,0.63,0.00}{#1}}
\newcommand{\BaseNTok}[1]{\textcolor[rgb]{0.00,0.00,0.81}{#1}}
\newcommand{\BuiltInTok}[1]{#1}
\newcommand{\CharTok}[1]{\textcolor[rgb]{0.31,0.60,0.02}{#1}}
\newcommand{\CommentTok}[1]{\textcolor[rgb]{0.56,0.35,0.01}{\textit{#1}}}
\newcommand{\CommentVarTok}[1]{\textcolor[rgb]{0.56,0.35,0.01}{\textbf{\textit{#1}}}}
\newcommand{\ConstantTok}[1]{\textcolor[rgb]{0.00,0.00,0.00}{#1}}
\newcommand{\ControlFlowTok}[1]{\textcolor[rgb]{0.13,0.29,0.53}{\textbf{#1}}}
\newcommand{\DataTypeTok}[1]{\textcolor[rgb]{0.13,0.29,0.53}{#1}}
\newcommand{\DecValTok}[1]{\textcolor[rgb]{0.00,0.00,0.81}{#1}}
\newcommand{\DocumentationTok}[1]{\textcolor[rgb]{0.56,0.35,0.01}{\textbf{\textit{#1}}}}
\newcommand{\ErrorTok}[1]{\textcolor[rgb]{0.64,0.00,0.00}{\textbf{#1}}}
\newcommand{\ExtensionTok}[1]{#1}
\newcommand{\FloatTok}[1]{\textcolor[rgb]{0.00,0.00,0.81}{#1}}
\newcommand{\FunctionTok}[1]{\textcolor[rgb]{0.00,0.00,0.00}{#1}}
\newcommand{\ImportTok}[1]{#1}
\newcommand{\InformationTok}[1]{\textcolor[rgb]{0.56,0.35,0.01}{\textbf{\textit{#1}}}}
\newcommand{\KeywordTok}[1]{\textcolor[rgb]{0.13,0.29,0.53}{\textbf{#1}}}
\newcommand{\NormalTok}[1]{#1}
\newcommand{\OperatorTok}[1]{\textcolor[rgb]{0.81,0.36,0.00}{\textbf{#1}}}
\newcommand{\OtherTok}[1]{\textcolor[rgb]{0.56,0.35,0.01}{#1}}
\newcommand{\PreprocessorTok}[1]{\textcolor[rgb]{0.56,0.35,0.01}{\textit{#1}}}
\newcommand{\RegionMarkerTok}[1]{#1}
\newcommand{\SpecialCharTok}[1]{\textcolor[rgb]{0.00,0.00,0.00}{#1}}
\newcommand{\SpecialStringTok}[1]{\textcolor[rgb]{0.31,0.60,0.02}{#1}}
\newcommand{\StringTok}[1]{\textcolor[rgb]{0.31,0.60,0.02}{#1}}
\newcommand{\VariableTok}[1]{\textcolor[rgb]{0.00,0.00,0.00}{#1}}
\newcommand{\VerbatimStringTok}[1]{\textcolor[rgb]{0.31,0.60,0.02}{#1}}
\newcommand{\WarningTok}[1]{\textcolor[rgb]{0.56,0.35,0.01}{\textbf{\textit{#1}}}}
\usepackage{graphicx}
\makeatletter
\def\maxwidth{\ifdim\Gin@nat@width>\linewidth\linewidth\else\Gin@nat@width\fi}
\def\maxheight{\ifdim\Gin@nat@height>\textheight\textheight\else\Gin@nat@height\fi}
\makeatother
% Scale images if necessary, so that they will not overflow the page
% margins by default, and it is still possible to overwrite the defaults
% using explicit options in \includegraphics[width, height, ...]{}
\setkeys{Gin}{width=\maxwidth,height=\maxheight,keepaspectratio}
% Set default figure placement to htbp
\makeatletter
\def\fps@figure{htbp}
\makeatother
\setlength{\emergencystretch}{3em} % prevent overfull lines
\providecommand{\tightlist}{%
  \setlength{\itemsep}{0pt}\setlength{\parskip}{0pt}}
\setcounter{secnumdepth}{-\maxdimen} % remove section numbering

\title{Homework4\_JKGWAS}
\author{Pabitra Joshi and Lindsey Kornowske}
\date{20 March 2021}

\begin{document}
\maketitle

{
\setcounter{tocdepth}{6}
\tableofcontents
}
\begin{figure*}[htbp]
\begin{center}
\includegraphics[width = 0.5\textwidth]{JKGWAS_logo.png}
\end{center}
\end{figure*}

\hypertarget{data-and-functions}{%
\section{Data and Functions}\label{data-and-functions}}

\begin{Shaded}
\begin{Highlighting}[]
\NormalTok{X =}\StringTok{ }\KeywordTok{read.csv}\NormalTok{(}\DataTypeTok{file =} \StringTok{"./../datasets/mdp\_numeric.txt"}\NormalTok{, }\DataTypeTok{header =} \OtherTok{TRUE}\NormalTok{, }\DataTypeTok{sep =}\StringTok{""}\NormalTok{);}
\NormalTok{y =}\StringTok{ }\KeywordTok{read.csv}\NormalTok{(}\DataTypeTok{file =} \StringTok{"./../datasets/CROPS545\_Phenotype.txt"}\NormalTok{, }\DataTypeTok{header =} \OtherTok{TRUE}\NormalTok{, }\DataTypeTok{sep =} \StringTok{""}\NormalTok{);}
\NormalTok{CV =}\StringTok{ }\KeywordTok{read.csv}\NormalTok{(}\DataTypeTok{file =} \StringTok{"./../datasets/CROPS545\_Covariates.txt"}\NormalTok{, }\DataTypeTok{header =} \OtherTok{TRUE}\NormalTok{, }\DataTypeTok{sep =} \StringTok{""}\NormalTok{);}
\NormalTok{SNP =}\StringTok{ }\KeywordTok{read.csv}\NormalTok{(}\DataTypeTok{file =} \StringTok{"./../datasets/mdp\_SNP\_information.txt"}\NormalTok{, }\DataTypeTok{header =} \OtherTok{TRUE}\NormalTok{, }\DataTypeTok{sep =} \StringTok{""}\NormalTok{);}

\CommentTok{\#library(JKGWAS)}
\end{Highlighting}
\end{Shaded}

\hypertarget{question-1-and-2}{%
\section{Question 1 and 2}\label{question-1-and-2}}

\textbf{(1) The package should contain at least three input: y, X , and C that are R objects of numeric data frame. Their dimensions are n by 1, n by m, and n by t corresponding to phenotype, genotype and covariate data, where n is number of individuals, m is number of markers, and t is number of covariates. The function should return probability values with dimension of 1 by m for the association tests between phenotype and markers. Markers are tested one at a time with covariates in C included as covariates (15 points).
(2) The package should perform PCA and incorporate PCs as cofactors for GWAS.  Your package should also automatically exclude the PCs that are in linear dependent to the covariates provided by users. (25 points).}

The JKGWAS Package contains four functions:

\par

\textbullet JKPCA takes genotype (X) data and covariate data (CV),
computes the PCA on X, then automatically removes PCs that are linearly
dependent to the CVs by method of comparing matrix rank. PCs are removed
from the matrix in succesion and those that do not change the rank by
removal are determined to be linearly independent because they do not
provide additional information.

\par

\textbullet JKGLM takes phenotype (y), genotype (X), covariate (CV), and
principal component (PC) inputs (ideally provided from JKPCA) and
returns p-values calculated for the association tests between the
phenotype and SNPs

\par

\textbullet JKQQ takes the pvalues from JKGLM and visualizes them by QQ
plot. Expected p-values of length m are simulated from the continuous
distribution.

\par

\textbullet JKManhattan visualizes the pvalues from JKGLM by Manhattan
plot. User input QTNs can also be visualized. The significance threshold
can be set, or it will default to Bonferoni correction for alpha = 0.05

\par

\hypertarget{question-3}{%
\section{Question 3}\label{question-3}}

\textbf{(3) Develop a user manual and tutorials. Name your package and create a logo. (20 points).}

The JKGWAS package is named for Pabitra Joshi and Lindsey Kornowske, the
label is displayed in Figure \ref{fig:logo}.

\begin{figure*}[htbp]
\begin{center}
\includegraphics[width = 0.3\textwidth]{JKGWAS_logo.png}
  \caption{JKGWAS Package Logo}
  \label{fig:logo}
\end{center}
\end{figure*}

\hypertarget{question-4}{%
\section{Question 4}\label{question-4}}

\textbf{(4) Perform GWAS on the data provided or your own data which must contain cofactors (15 points).}

\begin{Shaded}
\begin{Highlighting}[]
\CommentTok{\#\# Get Principal Components with JKPCA()}
\CommentTok{\#PC = JKPCA(X, CV, npc = 10);}

\CommentTok{\#\# Perform GWAS by GLM with JKGLM()}
\CommentTok{\#Pvals = JKGLM(X = X, y = y, CV = CV, PC = PC);}

\CommentTok{\#\# Visualize GWAS by QQ Plot with JKQQ()}
\CommentTok{\#JKQQ(Pvals);}

\CommentTok{\#\# Visualize GWAS by Manhattan Plot with JKManhattan()}
\CommentTok{\#JKManhattan(Pvals = Pvals, SNP = SNP, QTN = QTN);}
\end{Highlighting}
\end{Shaded}

\hypertarget{question-5}{%
\section{Question 5}\label{question-5}}

\textbf{(5) Demonstrate that your method is superior to the competing method (GWASbyCor) through simulation with at least 30 replicates (25 points).}

\begin{Shaded}
\begin{Highlighting}[]
\CommentTok{\# compareGWASnTimes = function(n = 100, X = X, qtn = 10) \{}
\CommentTok{\#   \# create array to store average for each iteration}
\CommentTok{\#   store.means = array();}
\CommentTok{\#   store.sig = array();}
\CommentTok{\#   store.total = array();}
\CommentTok{\#   }
\CommentTok{\#   for(i in 1:n)\{}
\CommentTok{\# \# first, simulate phenotype}
\CommentTok{\#    \# same parameters as Q2{-}4}
\CommentTok{\# G2P.sim = G2P(X=new.X, }
\CommentTok{\#              h2=.75,}
\CommentTok{\#              alpha=1,}
\CommentTok{\#              NQTN=qtn,}
\CommentTok{\#              distribution="norm");}
\CommentTok{\# \# get QTN positions in GD data}
\CommentTok{\# G2P.sim.qtn = G2P.sim$QTN.position;}
\CommentTok{\# }
\CommentTok{\# \#get array of pvals }
\CommentTok{\# p.list = GWASbyCor(new.X, G2P.sim$y);}
\CommentTok{\# \#sort array of pvals as increasing }
\CommentTok{\# p.index=sort(p.list, decreasing = FALSE);}
\CommentTok{\# p.max10=p.index[1:10];}
\CommentTok{\# }
\CommentTok{\# \#identify qtn pvals}
\CommentTok{\# qtn.p = p.list[ G2P.sim.qtn]}
\CommentTok{\# }
\CommentTok{\# }\AlertTok{\#\#\#}\CommentTok{ Question 3}
\CommentTok{\# \#find all qtn pvals in total list}
\CommentTok{\# truePos=intersect(p.max10,qtn.p);}
\CommentTok{\# \#find all significant pvals not in qtn list}
\CommentTok{\# falsePos=setdiff(p.max10, qtn.p);}
\CommentTok{\# \#save the number of true positives for this iteration}
\CommentTok{\# store.means[i] = length(truePos);}
\CommentTok{\# }
\CommentTok{\# }\AlertTok{\#\#\#}\CommentTok{ Question 4}
\CommentTok{\# \# get p value of 7th qtn}
\CommentTok{\# p.QTN7 = sort(qtn.p, decreasing = TRUE);}
\CommentTok{\# p.QTN7 = p.QTN7[7];}
\CommentTok{\# \# find all pvals in list that are more significant}
\CommentTok{\# p.below.qtn7 = p.index[p.index \textless{} p.QTN7];}
\CommentTok{\# \# save this number}
\CommentTok{\# store.sig[i] = length(p.below.qtn7)}
\CommentTok{\# \# total pvals without NAs}
\CommentTok{\# store.total[i] = length(p.index) }
\CommentTok{\# }
\CommentTok{\# }
\CommentTok{\#   \}}
\CommentTok{\#   \#get numeric count data}
\CommentTok{\#   store.means = as.numeric(as.character(store.means)); }
\CommentTok{\#   \#combine into single df for summary stats}
\CommentTok{\#   res = as.data.frame(cbind(store.means, store.sig, store.total));}
\CommentTok{\#   \#compute summary statistics}
\CommentTok{\#   means.result = res \%\textgreater{}\% summarize\_if(is.numeric, .funs = c("mean", "sd"));}
\CommentTok{\#   means.result;}
\CommentTok{\# \}}
\end{Highlighting}
\end{Shaded}

\hypertarget{extra-credit}{%
\section{Extra Credit}\label{extra-credit}}

\textbf{(6) Demonstrate that your package is better than BLINK C version (http://zzlab.net/blink) on either statistical power or speed (25 points). }

\end{document}
